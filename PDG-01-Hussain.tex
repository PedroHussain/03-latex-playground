% This is a template for doing homework assignments in LaTeX

\documentclass[a4paper, 12pt]{article} % This command is used to set the type of document you are working on such as an article, book, or presenation

\usepackage{amsmath, amssymb}  % This package allows the use of a large range of mathematical formula, commands, and symbols

\usepackage[top=3cm, bottom=2cm, left=2.5cm, right=2cm]{geometry}

\newcommand{\makemyblock}[3]{

    \begin{center}
        \Large \textbf{#1} \normalsize
    \end{center}

    \vspace{1cm}

%    \begin{table}[]
    \begin{flushleft}        
        \begin{tabular}{|l|l|}
            \hline
            \textbf{Name:}       & Pedro Hussain \\
            \hline
            \textbf{MatrikelNr:} & 7393841 \\
            \hline
            \textbf{Gruppe:}     & {#2} \\
            \hline
            \textbf{Blatt:}      & {#3} \\
            \hline
        \end{tabular}
    \end{flushleft}
    %    \end{table}

    \newcommand*{\QEDA}{\null\nobreak\hfill\ensuremath{\blacksquare}}%
    \newcommand*{\QEDB}{\null\nobreak\hfill\ensuremath{\square}}%

}



\begin{document}

\makemyblock{Übungen zur Einführung in partielle Differentialgleichungen}
{3}{01}
\vspace{1cm}

\begin{flushleft}

    \section*{Aufgabe 1a}

    \subsection*{Gegeben:}
    $u \in C^{\infty}(\mathbb{R}^n\setminus\{0\}) $, $n\ge 2$, mit
    $$
        u(x) := \begin{cases}
            \|x\|^{2-n} & n\ge 3 \\
            \log(\|x\|) & n=2
        \end{cases}
    $$

    \subsection*{Behauptung:}
    $\Delta u(x) = 0$ \, auf \, $\mathbb{R}^n\setminus\{0\}$

    \subsection*{Beweis:}
    \textbf{Fall 1:} $n = 2$.
    Für $i \in \{1,2\}$ gilt:
    $$
        \partial_i u = \frac{1}{\|x\|} \partial_i \|x\|
        = \frac{1}{\|x\|} \frac{x_i}{\|x\|}
        = \frac{x_i}{\|x\|^2}
    $$
    Daher
    $$
        \partial^2_i u = \frac{\|x\|^2-2x^2_i}{\|x\|^4}
    $$
    Summation über $i=1,2$ ergibt
    $$
        \Delta u(x) = \frac{2\|x\|^2 - 2\|x\|^2}{\|x\|^4}
        = 0 \quad \surd
    $$
    \textbf{Fall 1:} $n \ge 3$.
    Für $i \in \{1,...n\}$ gilt:
    $$
        \partial_i u = (2-n)\|x\|^{1-n}\frac{x_i}{\|x\|}
        = (2-n)\|x\|^{-n}x_i
    $$
    Also
    $$
        \partial^2_i u = (2-n) \left ((-n)\|x\|^{-n-1}\frac{x^2_i}{\|x\|}+\|x\|^{-n} \right )
        = (2-n) \|x\|^{-n} \left ( -\frac{n}{\|x\|^2} x^2_i + 1 \right )
    $$
    Summation über $i=1,...n$ liefert
    $$
        \Delta u(x) = (2-n) \|x\|^{-n} \left ( -\frac{n}{\|x\|^2} \|x\|^2 + n \right )
        = 0  \quad \surd
    $$
    \QEDB


    \section*{Aufgabe 1b}

    \subsection*{Gegeben:}
    $u:\mathbb{R}^n \times (0,\infty)\rightarrow \mathbb{R}$ mit
    $$
        u(x) := t^{-\frac{n}{2}} e^{-\frac{\|x\|^2}{4t}}
    $$

    \subsection*{Behauptung:}
    $\partial_t u - \Delta_x u = 0$ \, auf \, $\mathbb{R}^n \times (0,\infty)$

    \subsection*{Beweis:}
    Es ist $u \in C^\infty(\mathbb{R}^n\times (0,\infty))$.\\
    \textbf{Zu $\partial_t u$:} Es gilt
    $$
        \partial_t u = \partial_t \left (t^{-\frac{n}{2}} \right ) e^{-\frac{\|x\|^2}{4t}}
        + t^{-\frac{n}{2}} \partial_t \left ( e^{-\frac{\|x\|^2}{4t}} \right )
    $$

    $$
        \partial_t \left (t^{-\frac{n}{2}} \right ) = -\frac{n}{2} t^{-\frac{n}{2}-1}
        = -\frac{n}{2t} t^{-\frac{n}{2}}
    $$

    $$
        \partial_t \left (t^{-\frac{n}{2}} \right ) e^{-\frac{\|x\|^2}{4t}}
        = -\frac{n}{2t}u
    $$

    $$
        \partial_t \left (-\frac{\|x\|^2}{4t} \right ) = \frac{\|x\|^2}{4t^2}
    $$

    $$
        \partial_t \left (e^{-\frac{\|x\|^2}{4t}} \right )
        = e^{-\frac{\|x\|^2}{4t}} \frac{\|x\|^2}{4t^2}
    $$

    $$
        t^{-\frac{n}{2}} \partial_t \left ( e^{-\frac{\|x\|^2}{4t}} \right )
        = \frac{\|x\|^2}{4t^2} u
    $$
    Zusammengefasst
    $$
        \partial_t u = \left ( \frac{\|x\|^2}{4t^2} - \frac{n}{2t} \right ) u
    $$

    \textbf{Zu $\Delta_x u$:} Sei $i \in \{1,...,n\}$. Dann ist
    $$
        \partial_i u = t^{-\frac{n}{2}} \partial_i \left (e^{-\frac{\|x\|^2}{4t}} \right )
        = t^{-\frac{n}{2}} e^{-\frac{\|x\|^2}{4t}} \partial_i \left (-\frac{\|x\|^2}{4t} \right )
        = - \frac{x_i}{2t} u
    $$

    $$
        \partial^2_i u = - \frac{1}{2t} (u + x_i \partial_i u)
        = - \frac{1}{2t} (u - \frac{x^2_i}{2t} u)
        = - \frac{1}{2t} (1 - \frac{x^2_i}{2t}) u
    $$
    Summation über $i \in \{1,...,n\}$ ergibt
    $$
        \Delta_x u = - \frac{1}{2t} \left (n - \frac{\|x\|^2}{2t} \right ) u
        = \left (\frac{\|x\|^2}{4t^2}- \frac{n}{2t} \right ) u
    $$
    Daher folgt
    $$
        \partial_t u - \Delta_x u = 0
    $$
    \QEDB


    \section*{Aufgabe 2a}

    \subsection*{Gegeben:}
    $u \in C^2((a,b) \times (c,d))$ mit: $\partial_1\partial_2u = 0$ auf $(a,b) \times (c,d)
    $

    \subsection*{Behauptung:}
    Es existieren $f \in C^2((a,b))$, $g \in C^2((c,d))$ mit $u(x_1,x_2) = f(x_1)+g(x_2)$

    \subsection*{Beweis:}
    Seien $\overline{x}_1 \in (a,b)$ und $\overline{x}_2 \in (c,d)$ beliebig aber fest. \\
    Wegen $\partial_1\partial_2U=0$ muss für jedes $x_2 \in (c,d)$ die Funktion
    $$
        \partial_2u(\cdot,x_2): (a,b) \rightarrow \mathbb{R}
    $$
    konstant sein, also
    $$
        \partial_2u(x_1,x_2) = \partial_2u(\overline{x}_1,x_2) \quad \forall x_1 \in (a,b)
    $$
    Definiere $\varphi:(c,d)\rightarrow \mathbb{R}$ durch $\varphi(x_2)=\partial_2u(\overline{x}_1,x_2)$ und sei $\Phi$ eine Stammfunktion von $\varphi$. \\
    Dann ist $\Phi \in C^2((c,d))$. Für $x_1 \in (a,b), x_2 \in (c,d)$ gilt
    $$
        u(x_1,x_2)-u(x_1,\overline{x}_2)
        = \int_{\overline{x}_2}^{x_2} \partial_2 u(x_1,\xi_2)d\xi_2
        = \int_{\overline{x}_2}^{x_2} \varphi(\xi_2)d\xi_2
        = \Phi(x_2)-\Phi(\overline{x}_2)
    $$
    Also
    $$
        x(x_1,x_2) = u(x_1,\overline{x}_2) + \Phi(x_2) - \Phi(\overline{x}_2)
    $$
    Definiere $f:(a,b)\rightarrow\mathbb{R}$, $f(x_1) := u(x_1,\overline{x}_2)$,
    und $g:(c,d)\rightarrow\mathbb{R}$, $g(x_2) := \Phi(x_2) - \Phi(\overline{x}_2)$.
    Dann ist $f \in C^2((a,b))$, $g \in C^2((c,d))$ und für $x_1 \in (a,b)$, $x_2 \in (c,d)$ gilt
    $$
        u(x_1,x_2)=f(x_1)+g(x_2)
    $$

    \QEDB


    \section*{Aufgabe 2b}

    \subsection*{Gegeben:}
    $u \in C^2(\mathbb{R}^2)$ mit $\partial^2_1u - \partial^2_2u = 0$

    \subsection*{Behauptung:}
    Es existieren $f, g \in C^2(\mathbb{R})$ mit $u(x_1,x_2) = f(x_1+x_2)+g(x_1-x_2)$

    \subsection*{Beweis:}
    Definiere $v \in C^2(\mathbb{R}^2)$,
    $v(y_1,y_2) := u(\frac{y_1+y_2}{2},\frac{y_1-y_2}{2})$. Dann gilt
    $$
        \partial_2 v(y_1,y_2) = \frac{1}{2} \left ( \partial_1 u(\frac{y_1+y_2}{2},\frac{y_1-y_2}{2})
        - \partial_2 u(\frac{y_1+y_2}{2},\frac{y_1-y_2}{2}) \right )
    $$

    $$
        \frac{\partial}{\partial_{y_1}} \left (\partial_1 u(\frac{y_1+y_2}{2},\frac{y_1-y_2}{2}) \right )
        = \frac{1}{2} \left ( \partial^2_{11} u(\frac{y_1+y_2}{2},\frac{y_1-y_2}{2})
        + \partial^2_{21} u(\frac{y_1+y_2}{2},\frac{y_1-y_2}{2}) \right )
    $$

    $$
        \frac{\partial}{\partial_{y_1}} \left (\partial_2 u(\frac{y_1+y_2}{2},\frac{y_1-y_2}{2}) \right )
        = \frac{1}{2} \left ( \partial^2_{12} u(\frac{y_1+y_2}{2},\frac{y_1-y_2}{2})
        + \partial^2_{22} u(\frac{y_1+y_2}{2},\frac{y_1-y_2}{2}) \right )
    $$
    Es folgt
    $$
        \partial_1\partial_2 v(y_1,y_2) = \frac{1}{4} \left ( \partial^2_{11} u(\frac{y_1+y_2}{2},\frac{y_1-y_2}{2})
        - \partial^2_{22} u(\frac{y_1+y_2}{2},\frac{y_1-y_2}{2}) \right )
        = 0
    $$
    Nach \textbf{Aufgabe 2a} mit $(a,b)=(c,d)= \mathbb{R}$ existieren daher
    Funktionen $f, g \in C^2(\mathbb{R})$ mit
    $$
        v(y_1,y_2)= f(y_1)+g(y_2)
    $$
    Also folgt mit $x_1=\frac{y_1+y_2}{2}, x_2=\frac{y_1-y_2}{2}$
    $$
        u(x_1,x_2)=v(x_1+x_2,x_1-x_2)=f(x_1+x_2)+g(x_1-x_2)
    $$

    \QEDB

    \section*{Aufgabe 3}

    \subsection*{Gegeben:}
    $U \subset \mathbb{R}^n$ offen, $f \in C^\infty(U)$, $F \in C^\infty(U, \mathbb{R}^3)$.

    \subsection*{Behauptung:}
    (a) $\text{div} (\text{grad} f) = \Delta f$\\
    (b) Für $n=3$: $\text{rot} ( \text{grad} \, f) = 0$\\
    (c) $\text{div}(\text{rot}F) = 0$

    \subsection*{Beweis:}
    \underline{Zu (a):} Es ist
    $$
        \text{div} (\text{grad} f)
        = \text{div} (\partial_1 f,..., \partial_n f) )
        = \partial_1(\partial_1)f + ... + \partial_n(\partial_n)f
        = \partial^2_1 f + ... + \partial^2_n f = \Delta f  \quad \surd
    $$

    \hfill\\

    \underline{Zu (b):} Es ist
    $$
        \text{rot} ( \text{grad} \, f)
        = \text{rot} \begin{bmatrix}
            \partial_1 f \\
            \partial_2 f \\
            \partial_3 f
        \end{bmatrix}
        = \begin{bmatrix}
            \partial_2 \partial_3 f - \partial_3 \partial_2 f       \\
            - ( \partial_1 \partial_3 f - \partial_3 \partial_1 f ) \\
            \partial_1 \partial_2 f - \partial_2 \partial_1 f
        \end{bmatrix}
        = \begin{bmatrix}
            0 \\
            0 \\
            0
        \end{bmatrix}  \quad \surd
    $$
    Lezteres gilt wegen der Vertauschbarkeit der partiellen Ableitungen
    (Schwartscher Satz).\\

    \hfill\\

    \underline{Zu (c):} Es ist
    $$
        \text{div}(\text{rot}F)
        = \text{div} \begin{bmatrix}
            \partial_2 F_3 - \partial_3 F_2    \\
            -(\partial_1 F_3 - \partial_2 F_1) \\
            \partial_1 F_2 - \partial_2 F_1
        \end{bmatrix}
        = \partial_1 \partial_2 F_3 - \partial_1 \partial_3 F_2
        - \partial_2 \partial_1 F_3 + \partial_2 \partial_2 F_1
        + \partial_3 \partial_1 F_2 - \partial_3 \partial_2 F_1
    $$
    Die rechte Seite verschwindet, wieder aufgrund des Schwartz'schen Satzes.$\quad \surd$

    \QEDB

    \section*{Aufgabe 4}

    \subsection*{Gegeben:}
    $\alpha, \beta \in \mathbb{N}^n_0$.

    \subsection*{Behauptung:}
    $$
        \partial^\alpha x^\beta = \begin{cases}
            \frac{\alpha !}{(\alpha - \beta)!} x^{\alpha - \beta} & \text{falls } \beta \le \alpha \\
            0                                                     & \text{sonst}
        \end{cases}
    $$

    \subsection*{Beweis:}
    Induktion über $n$.

    \hfill\\

    \underline{$n=1$:} Es ist also $\alpha, \beta \in \mathbb{N}_0$. Daher gilt die Behauptung nach den eindimensionalen Differationsregeln.

    \hfill\\

    \underline{$n \rightarrow n+1$:} Gelte die Behauptung für ein $n \in \mathbb{N}$. Seien $\alpha, \beta \in \mathbb{N}^{n+1}_0$. Setze $\alpha = (\alpha_1,\underline{\alpha})$ und $\alpha = (\beta_1,\underline{\beta})$
    $$
        \partial^\alpha x^\beta = \partial^{\alpha_1} x^{\beta_1}
        = \begin{cases}
            \frac{\alpha_1 !}{(\alpha - \beta)!} x^{\alpha - \beta} & \text{falls } \beta \le \alpha \\
            0                                                       & \text{sonst}
        \end{cases}
    $$




\end{flushleft}

\end{document}